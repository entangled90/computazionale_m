\input{header.tex}

\begin{document}

\input{title.tex}


\section{Modello di Ising}
Il modello è stato implementato su un reticolo bidimensione di larghezza N e inizializzando gli spin al valore $\pm 1 $  con uguale probabilità.
Sono stati implementati due algoritmi: Metropolis e Swendsen-Wang.\\

\subsection{Algoritmo Metropolis}
\subsubsection{Termalizzazione}
Possiamo valutare quando avviene la corretta termalizzazione del sistema analizzando l'andamento della magnetizzazione e dell'energia in funzione del tempo markoviano.
Di seguito è riportato l'andamento dell'energia durante la termalizzazione a due diverse temperature, di cui una è la temperatura critica:\\
\begin{figure}[h]
\includegraphics[scale=0.35]{metropolis/en_therm.png}
\includegraphics[scale=0.35]{metropolis/en_therm_crit.png}
\caption{Energia a $\beta=0.3$ e $\beta=0.43$}
\end{figure}
\\
Di seguito la magnetizzazione:\\
\begin{figure}[h]
\includegraphics[scale=0.35]{metropolis/mag_therm.png}
\includegraphics[scale=0.35]{metropolis/mag_therm_crit.png}
\caption{Magnetizzazione a $\beta=0.3$ e $\beta=0.43$}
\end{figure}
Come si può vedere dal grafico relativo alla magnetizzazione a $\beta=0.43$, vicino al punto critico il tempo di termalizzazione cresce di molto. Esso rimane comunque intorno a 100 alla temperatura critica.
Dopo l'analisi di questi dati si è deciso di fissare il tempo di termalizzazione a 1000 passi temporali.

\subsubsection{Autocorrelazione fra le configurazioni}




\end{document}
