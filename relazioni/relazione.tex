\input{header.tex}

\begin{document}

\input{title.tex}
\tableofcontents 
\newpage
\section{Modello di Ising}
Il modello è stato implementato su un reticolo bidimensione di larghezza N e inizializzando gli spin al valore $\pm 1 $  con uguale probabilità.
Sono stati implementati due algoritmi: Metropolis e Swendsen-Wang.\\

\subsection{Termalizzazione}
Possiamo valutare quando avviene la corretta termalizzazione del sistema analizzando l'andamento della magnetizzazione e dell'energia in funzione del tempo markoviano.
Di seguito è riportato l'andamento dell'energia e della magnetizzazione durante la termalizzazione a due diverse temperature.
\subsubsection*{Metropolis}
\begin{figure}[h]
\includegraphics[scale=0.35]{metropolis/en_therm.png}
\includegraphics[scale=0.35]{metropolis/en_therm_crit.png}
\caption{Energia (Metropolis) a $\beta=0.3$ e $\beta=0.43$}
\end{figure}
\begin{figure}[h]
\includegraphics[scale=0.35]{metropolis/mag_therm.png}
\includegraphics[scale=0.35]{metropolis/mag_therm_crit.png}
\caption{Magnetizzazione  (Metropolis) a $\beta=0.3$ e $\beta=0.43$}
\end{figure}
Come si può vedere dal grafico relativo alla magnetizzazione a $\beta=0.43$, vicino al punto critico il tempo di termalizzazione cresce di molto. Esso rimane comunque intorno a 100 alla temperatura critica.
Dopo l'analisi di questi dati si è deciso di fissare il tempo di termalizzazione a 1000 passi temporali.
\subsubsection*{Swendsen-Wang}
\begin{figure}[h]
\includegraphics[scale=0.35]{sw/en_therm0-3.png}
\includegraphics[scale=0.35]{sw/en_therm0-43.png}
\caption{Energia (Swendsen-Wang) a $\beta=0.3$ e $\beta=0.43$}
\end{figure}

\begin{figure}[h]
\includegraphics[scale=0.35]{sw/mag_therm0-3.png}
\includegraphics[scale=0.35]{sw/mag_therm0-43.png}
\caption{Magnetizzazione  (Swendsen-Wang) a $\beta=0.3$ e $\beta=0.43$}
\end{figure}
Anche con questo algoritmo, il tempo di termalizzazione è molto minore dei 1000 passi utilizzati per far termalizzare il sistema nel programma.
La sostanziale differenza che si vede nei grafici relativi alla magnetizzazione vicino al punto critico è dovuta all'inefficienza dell'algoritmo Metropolis nell'estrarre configurazioni scorrelate vicino al punto critico, fenomeno solitamente chiamato \emph{critical slowing down}.
Esso si manifesterà più chiaramente nello studio dell'autocorrelazione delle configurazioni.


\subsection{Autocorrelazione fra le configurazioni}
Come in ogni simulazione Monte Carlo, è necessario studiare l'autocorrelazione delle configurazione estratte attraverso l'algoritmo in modo da stimare l'efficienza dell'algoritmo ad estrarre configurazioni
indipendenti. Si confronti l'autocorrelazione dell'energia in funzione della temperatura tra i due algoritmi:
\begin{figure}[h]
\includegraphics[scale=0.35]{sw/mag_therm0-3.png}
\includegraphics[scale=0.35]{sw/en_tau.png}
\caption{Autocorrelazione dell'energia. A sinistra Metropolis, a Destra Swendsen-Wang}
\end{figure}



\end{document}
